%\documentclass[a1paper,landscape,showframe,fontscale=.42]{baposter}

%%THIS are the max size given by the COMPLENET website and is bigger than A1!:
%%\documentclass[paperwidth=42in, paperheight=42in,landscape,showframe,fontscale=.42]{baposter}
%%\documentclass[paperwidth=42in, paperheight=33.1in,landscape,showframe,fontscale=.42]{baposter}
\documentclass[a1paper,portrait,showframe,fontscale=.46]{baposter}

%%%%lualatex on
%\usepackage{luatextra}
\usepackage{fontspec}
%Ligatures={Contextual, Common, Historical, Rare, Discretionary}
\setmainfont[Mapping=tex-text]{Linux Libertine O}


\usepackage{enumerate}
\usepackage[english]{babel}
\usepackage{graphicx} %to insert pictures
\usepackage{color} %to set colors
\usepackage{algorithm,algorithmicx,algpseudocode}
\usepackage{mathtools}
\usepackage{latexsym}
\usepackage{caption}
\usepackage{multicol}
\usepackage{array}

\usepackage{float}
\usepackage{booktabs}
\algnewcommand\And{\textbf{and}}

\DeclarePairedDelimiter\abs{\lvert}{\rvert}%
\DeclarePairedDelimiter\norm{\lVert}{\rVert}%

\newcommand{\specialcell}[2][c]{%
	\begin{tabular}[#1]{@{}c@{}}#2\end{tabular}}


		\makeatletter
		\let\oldabs\abs
		\def\abs{\@ifstar{\oldabs}{\oldabs*}}
		\let\oldnorm\norm
		\def\norm{\@ifstar{\oldnorm}{\oldnorm*}}
		\makeatother


%\usepackage[top=1.5cm,bottom=2cm,left=2.5cm,right=2.5cm]{geometry}
%\linespread{1.5}\selectfont



		\author{Simon Carrignon}
		\definecolor{bordercol}{RGB}{255,255,255}

		\definecolor{headercol1}{RGB}{142,161,42}
		\definecolor{epnetcol}{RGB}{142,161,42}
		\definecolor{headercol2}{RGB}{255,255,255}
		\definecolor{headerfontcol}{RGB}{78,78,78}
		\definecolor{boxcolor}{RGB}{255,255,255}
		\definecolor{emphcol}{RGB}{106,105,180}

%%% Save space in lists. Use this after the opening of the list %%%%%%%%%%%%%%%%
		\newcommand{\compresslist}{
		\setlength{\itemsep}{1pt}
			\setlength{\parskip}{0pt}
			\setlength{\parsep}{0pt}
		}
		\newcommand{\coloremph}[1]{
			\textcolor{emphcol}{\bf#1}
		}


		\begin{document}

		\begin{poster}{
				borderColor=bordercol,
				headerColorOne=headercol1,
				headerColorTwo=headercol2,
				headerFontColor=headerfontcol,
	% Only simple background color used, no shading, so boxColorTwo isn't necessary
				boxColorOne=boxcolor,
				headershape=roundedright,
				headerfont=\Large\sf\bf,
				textborder=rectangle,
				headerborder=open,
				background=plain,
				bgColorOne=white,
				boxshade=plain,
				eyecatcher,
				columns=2
			}
			{
			}
			{
				Mis cosillas
			}
			{
				Juan Moros, 
				Juan Moros,
				and Juan Moros\\
				{\small imorer@ffn.ub.es \& \small simon.carrignon@bsc.es}
			}
			{
				\setlength\fboxsep{0pt}
				\setlength\fboxrule{0.5pt}
				\begin{minipage}{14em}
			%\vspace*{\stretch{1}}
					\includegraphics[height=8em]{logos/epnetLogo2.png}
			%\vspace*{\stretch{1}}
			%\includegraphics[angle=90,width=2.5em]{MemoireLophiss/images/logo_p7_large}
				\end{minipage}
			}

			\headerbox{Introduction}{name=introduction,column=0,row=0}{
				Cultural change comprises  processes that modify spread of information by social interaction within a population~\cite{boyd_origin_2005}. Numerous social scientists are using an evolutionary framework to model this~\cite{henrich_evolution_2003}.

				Here we follow this trend to study the dynamics of an exchange based economy. This economy is a social activity depending on particular cultural traits: the value attributed to goods used to trade during the exchange. Multiple cultural parameters could influence the way those values evolve through space and time leading to different dynamics.

				In this study we focus on how the topology of the cultural network impacts the dynamics of such a trade based economy. To do so we start from a system where a mechanism of cultural transmission, biased toward the success of the individual, allows the agents to efficiently modify the value they attribute to each good in order that everyone can gather all the good they doesn't produce. The cultural networks allow all the producers of one good to know the economical success of the other producers of the same good and to copy their strategies.
				
				We propose an experimental setup that allow us to change the topology of this cultural network and to look at how the trade dynamics are affected by such changes. 
			}
			\headerbox{Model}{name=ud,column=1,row=0}{

				To explore how cultural network topologies influence the co-evolution between trade and cultural change, we developed a simple framework where the different agents produce and trade goods. The model is composed of a population $Pop$ of $m$ agents. Each agent $i$ is defined by 2 vectors $Q^i$ and $V^i$ of size $n$. $Q^i$ stores the quantity of each good owned by $i$ and $V^i$ represents the price estimated by $i$ for each of the $n$ good.
%\begin{algorithm}[H]
%	\scriptsize
%\caption{Model}
%\label{algo:complete}
%	\begin{algorithmic}[1]
%	\State INITIALIZATION: 
%		\For{$i \in \#Pop$} \Comment{Initialize the agent with no goods and a random value vector}
%				\State $Q^i = (0, \cdots, 0)$
%				\State $V^i = (v^i_0, \cdots, v^i_n)$ \Comment{The values of $v^i_j$ are selected randomly}
%		\EndFor
%
%	\State SIMULATION:
%		\Loop{$~step \in TimeSteps$}
%			\For{$i \in Pop$}
%				\State $Production(Q^i)$
%			\EndFor
%			\For{$i \in Pop$}
%				\For{$j \in Pop$}
%					\State $TradeProcess(V^i,Q^i,V^j,Q^j)$
%				\EndFor		
%			\EndFor
%			\For{$i \in Pop$}
%				\State $ConsumeGoods(Q^i)$ \Comment{All goods are consumed}
%				\If{$ (step \mod CulturalStep) = 0$}	
%					\State $CulturalTransmission(V)$
%					\State $Innovation(V^i)$
%				\EndIf
%			\EndFor
%		\EndLoop
%\end{algorithmic}
%\end{algorithm}
				Given the prices attributed by the agents for each good ($V^i$), an exchange is made or not using the trade network (in green in the Figure~\ref{fig:feedbackSchema}). Given the quantities ($Q^i$) gathered, a score reflecting the ``economic success'' of each agent is attributed. Using this score, agents use their cultural network (in blue in the Figure~\ref{fig:feedbackSchema}) to update the value attributed to each good $V^i$.

				\begin{figure}[H]
					\centering
					\includegraphics[trim={2cm 6cm 2cm 5cm},clip,width=8cm]{img/trade-cultural.png}
					\caption{ {\small Illustration of the interaction between the Trade network and the Cultural networks}}
					\label{fig:feedbackSchema}
				\end{figure}		


		%    We first compare the impact of different $CulturalTransmission$ mechanism on the distribution of frequencies of traits (the belief about the price of each goods). 
		%    \begin{figure}[H]
		%	\centering
		%	\setlength{\columnseprule}{0pt}
		%	\begin{multicols}{2}
		%	    \includegraphics[width=5.2cm]{img/2SetupDistribA.pdf} 
		%	    \caption{Comparaison of the distribution of frequencies between the neutral and the trade model.}
		%	    \label{fig:2setDi}
		%	\end{multicols}
		%    \end{figure}
		%    \vspace{-.8cm}
		%    The figure~\ref{fig:2setDi} shows that when $CulturalTransmission$ is neutral (agents randomly copy prices) the distribution follow the well know power law \cite{bentley_random_2004} but when transmission is not neutral but biased by the economical success of the agents, the power law disappear.

			}




			\headerbox{Simulations \& Results}{name=res1,column=0,span=2,below=ud}{
				\begin{multicols}{2}
					\subsection*{Simulation}
					In all the simulations we use a population of $m=600$ agents and $n=3$ goods. A penalty of $1$ is given to the agents unable to exchange their good with one of the other goods. If the exchange is made, the penalty is reduced if the quantity gathered ($Q^i$) is closer to an optimal value $O^i$ shared by all the agents. During one timestep, the agents exchange their good 10 times before updating the values they attributes to prices. The score of the agents is given by the sum of the penalties.

					\subsection*{Fully connected Cultural Network}
					We first carry out simulations in which cultural networks are complete \emph{i.e.} every agent knows the strategy of the every producers of its own good. 					
					\begin{figure}[H]
						\centering
						\begin{multicols}{2}
							\includegraphics[width=5cm]{img/full.pdf} 
							\caption{Evolution of the score of the agents in a setup with 600 agents and 3 goods trading and exchanging their strategies during 10000 timesteps.}%%
							\label{fig:scoreEvol}
						\end{multicols}
					\end{figure}
					\vspace{-1cm}
					When the cultural network is fully connected, all the mean score of the agent converge to a value around 3. It means that during the 10 exchange they make, in the worst case there is 3 exchanges during which they are not able to exchange one of the goods.
					\subsection*{Influence of Average Distance ($L$) and Average Degree ($\left\langle k\right\rangle$)}
					\begin{multicols}{2}
						To test the influence of the topology of cultural networks we build several networks with the same average distance $L$ and different average degree $\left\langle k\right\rangle$. 
						
						To reach those values, we create chain or ring lattices of $v$ neighbours and then we rewire other lattices with $v'<v$ until the former network's $L$ is achieved.  
						\columnbreak
						\vspace{1cm}
						\begin{center}
							\large
							\begin{tabular}{l|ccc}
								& $\left\langle k\right\rangle_1$	 	& \dots & $\left\langle k\right\rangle_n$		\\\hline
								$L_1$	& $G_{11}$	& 	& $G_{1n}$	\\	
								\dots	&		&\dots	&		\\
								$L_m$	& $G_{m1}$	& 	& $G_{mn}$	\\	
							\end{tabular}
						\end{center}
					\end{multicols}

					\columnbreak 

					The next table illustrates some of the topologies we designed for networks with 200 agents and for some couple of values $\{L,\left\langle k\right\rangle\}$.
					\begin{center}
						\begin{tabular}{m{1.2cm}m{2cm}m{2cm}}
							&$\left\langle k\right\rangle=8$ & $\left\langle k\right\rangle=4$\\
							$L\approx17$&
							\includegraphics[width=2cm]{img/g02.pdf}&
							\includegraphics[width=2cm]{img/g00.pdf}\\
							$L\approx4$&
							\includegraphics[width=2.5cm]{img/g42.pdf}&
							\includegraphics[width=2.5cm]{img/g40.pdf}\\
						\end{tabular}
					\end{center}
					\subsection*{Results}
					The average distance $L$ has been proven to be the key property for the simulations not only to reach the equilibrium faster, but also to enhance performances for the agents. For equal values of $L$, there is no clear influence of the density in either aspects. Nonetheless, the best performance is worse in these more realistic networks than in the fully connected experiment. 
					\begin{multicols}{2}
					\begin{figure}[H]
						\center
						\includegraphics[width=7cm]{img/provResults2.pdf}
						\caption{ \small
						Behaviour of the simulations with different topologies.}
						\label{fig:resultsTs}
					\end{figure}
					\columnbreak
						\begin{figure}[H]
						\center
						\includegraphics[width=5cm]{img/L_vs_t10.pdf}
						\caption{ \small
						Time of convergence as a function of $L$.}
						\label{fig:resultsConv}
					\end{figure}
					\end{multicols}


				\end{multicols}
			}
			\headerbox{Concluding Remarks}{name=conclusion,column=0,below=res1}{
				Trade dynamics and cultural mechanisms are often studied separately. A model of cultural evolution allows us to bring together those two aspects and to perform quantitative analysis of the influence of both side on the general dynamic of the system.

				To illustrate the importance of the interaction between cultural and economic mechanism, we show with this study how the some particular properties of the cultural network influence the speed of convergence of the trade mechanisms. This strongly suggest that when studying economic dynamics one cannot put aside the cultural mechanism underlying the economy.

				In following studies we want to measure how the properties shown here can improve the economic resilience of a system under unstable conditions.
			}


			\headerbox{References}{name=references,column=1,below=res1}{
				\scriptsize
				\renewcommand{\refname}{\vspace{-0.5em}}
				\bibliographystyle{unsrt}
				\bibliography{biblio}
			}
			\headerbox{Acknowledgements}{name=acknowledgements,column=1,below=references}{

				Funding for this work was provided by the ERC Advanced Grant EPNet (340828).
				\begin{center}
					\begin{tabular}{m{2cm}m{2cm}m{2cm}}
						\includegraphics[width=2cm]{logos/bscLogo.jpg}&
						\includegraphics[width=1.5cm]{logos/LOGO-ERC.jpg}&
						\includegraphics[width=2cm]{logos/logoUB.jpg}
					\end{tabular}
				\end{center}
			} 

		\end{poster}

		\end{document}

